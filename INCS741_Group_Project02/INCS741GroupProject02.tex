%%%%%%%%%%%%%%%%%%%%%%%%%%%%%%%%%%%%%%%%%
% Journal Article
% LaTeX Template
% Version 1.4 (15/5/16)
%
% This template has been downloaded from:
% http://www.LaTeXTemplates.com
%
% Original author:
% Frits Wenneker (http://www.howtotex.com) with extensive modifications by
% Vel (vel@LaTeXTemplates.com)
%
% License:
% CC BY-NC-SA 3.0 (http://creativecommons.org/licenses/by-nc-sa/3.0/)
%
%%%%%%%%%%%%%%%%%%%%%%%%%%%%%%%%%%%%%%%%%

%----------------------------------------------------------------------------------------
%	PACKAGES AND OTHER DOCUMENT CONFIGURATIONS
%----------------------------------------------------------------------------------------

\documentclass[twoside,twocolumn]{article}


\usepackage[sc]{mathpazo} % Use the Palatino font
\usepackage[T1]{fontenc} % Use 8-bit encoding that has 256 glyphs
\linespread{1.05} % Line spacing - Palatino needs more space between lines
\usepackage{microtype} % Slightly tweak font spacing for aesthetics

\usepackage{amsmath}
\usepackage{algorithm}
\usepackage[noend]{algpseudocode}

\usepackage[english]{babel} % Language hyphenation and typographical rules

\usepackage[hmarginratio=1:1,top=32mm,columnsep=20pt]{geometry} % Document margins
\usepackage[hang, small,labelfont=bf,up,textfont=it,up]{caption} % Custom captions under/above floats in tables or figures
\usepackage{booktabs} % Horizontal rules in tables

\usepackage{lettrine} % The lettrine is the first enlarged letter at the beginning of the text

\usepackage{enumitem} % Customized lists
\setlist[itemize]{noitemsep} % Make itemize lists more compact

\usepackage{abstract} % Allows abstract customization
\renewcommand{\abstractnamefont}{\normalfont\bfseries} % Set the "Abstract" text to bold
\renewcommand{\abstracttextfont}{\normalfont\small\itshape} % Set the abstract itself to small italic text

\usepackage{titlesec} % Allows customization of titles
\renewcommand\thesection{\Roman{section}} % Roman numerals for the sections
\renewcommand\thesubsection{\roman{subsection}} % roman numerals for subsections
\titleformat{\section}[block]{\large\scshape\centering}{\thesection.}{1em}{} % Change the look of the section titles
\titleformat{\subsection}[block]{\large}{\thesubsection.}{1em}{} % Change the look of the section titles
\titlespacing*{\section} {0pt}{3.5ex plus 1ex minus .2ex}{2.3ex plus .2ex}
\usepackage{fancyhdr} % Headers and footers
\pagestyle{fancy} % All pages have headers and footers
\fancyhead{} % Blank out the default header
\fancyfoot{} % Blank out the default footer
\fancyhead[C]{INCS741 Group 10 Project$\bullet$ Feb 2022} % Custom header text
\fancyfoot[RO,LE]{\thepage} % Custom footer text

\usepackage{titling} % Customizing the title section

\usepackage{hyperref} % For hyperlinks in the PDF

\usepackage{graphicx}

\usepackage{caption}
\usepackage{float} 
\usepackage{subcaption}
\usepackage{amsmath}
\usepackage{algpseudocode}




%----------------------------------------------------------------------------------------
%	TITLE SECTION
%----------------------------------------------------------------------------------------

\setlength{\droptitle}{-4\baselineskip} % Move the title up
\setlength{\textfloatsep}{1.5pt}

\pretitle{\begin{center}\Huge\bfseries} % Article title formatting
\posttitle{\end{center}} % Article title closing formatting
\title{ROW TRANSPOSITION CIPHER IMPLEMENTATION} % Article title
%\author{%
%\textsc{Saba Mohammadi }\thanks{A thank you or further information} \\[1ex] % Your name
%\normalsize NYIT \\ % Your institution
%\normalsize \href{mailto:john@smith.com}{smoham87@nyit.edu} % Your email address
%\and % Uncomment if 2 authors are required, duplicate these 4 lines if more
%\textsc{Jane Smith}\thanks{Corresponding author} \\[1ex] % Second author's name
%\normalsize University of Utah \\ % Second author's institution
%\normalsize \href{mailto:jane@smith.com}{jane@smith.com} % Second author's email address
%}

\author{ \\ \footnotesize Shiyun hu  \\ \footnotesize Student ID 1298223\\ \footnotesize New York Institute of Technology \\ \footnotesize shu10@nyit.edu \and \\\footnotesize Jun Jie Wu \\ \footnotesize Student ID 1298381 \\ \footnotesize New York Institute of Technology \\ \footnotesize jwu72@nyit.edu  \and \\ \footnotesize Leonardo Amorim de Lemos  \\ \footnotesize Student ID 1292678\\ \footnotesize New York Institute of Technology\\ \footnotesize lamori01@nyit.edu \\}

\date{\today} % Leave empty to omit a date
\renewcommand{\maketitlehookd}{%

\begin{abstract}
\noindent  % Dummy abstract text - replace \blindtext with your abstract text
The cipher appeared thousands of years ago but was heavily exploited from the 2nd World War. In our group project, we have learn one of the cryptography method in the transposition ciphers filed called Row Transposition cipher. Our group implements the Row Transposition cipher's encryption and decryption algorithm by using Java programming language. 

\end{abstract}
}

%----------------------------------------------------------------------------------------
\setlength{\intextsep}{9.5pt plus 1pt minus 2pt}
\begin{document}

% Print the title
\maketitle

%----------------------------------------------------------------------------------------
%	ARTICLE CONTENTS
%----------------------------------------------------------------------------------------

\section{Introduction}

\lettrine[nindent=2em,lines=1] {T}he cipher appeared thousands of years ago but was heavily exploited from the 2nd World War in which information exchanged between enemies, even encrypted, was intercepted and deciphered. There are two widely explored types of encryptions (transposition and substitution) that, when worked independently, can be solved quickly. Still, protocols have been created and improved that fundamentally use these two techniques repetitively. By definition, transposition cipher changes the plaintext order and rearranges to get ciphertext. In this group project, we used “Row Transposition Cipher,” where you write your plaintext in rows of fixed length (key size), and we write by columns in key order. We can use the procedure to enhance the complexity of a more complex cipher-text.


\section{Encryption Implementation}
To implement the row transposition encryption, we utilize the key as a sequence to switch the columns in a two-dimension matrix to form a row transposition matrix(Figure 3).

Take the key 'NYITV' as an example (Figure 1), the algorithm uses the 26 English letters to find the number sequence '14023'. Then, the algorithm arranges the columns by the order of this number sequence. The encryption algorithm writes letters of message out in rows over a specified number of columns which equals the key length '5' (Figure 2). Then, reorder columns in the matrix (Figure 3). \\



\begin{figure}[!ht]
  \centering
  \includegraphics[scale=0.3]{./Graphs/Figure1.1.png}
  \caption{Task1-Encryption Sequence Order}
  \label{fig:testfig1}
\end{figure}

The sequence reordering is '20341'(Figure1) for decryption in reading the columns of 2D Matrix, which transfers from the key order '14023'. For instance, the 0 column read 1st and it is in the RowTranspositionMatrix column 2. In Figure 2, we read columns by following the reordering sequence '20341' in the matrix to get the encryption message.


Regards for the current assignment, the empty space would be replaced by the capital letter 'X' to append the rows to form the ciphertext.\\

\begin{figure}[H]
  \centering
  \includegraphics[scale=0.4]{./Graphs/Figure1.2.png}
  \caption{Task1-Row Transcription Matrix}
  \label{fig:testfig1}
\end{figure}


The following figure 3 displays the encrypted message.

\begin{figure}[H]
  \centering
  \includegraphics[scale=0.75]{./Graphs/Figure1.4.png}
  \caption{Task1-Encrypted Message}
  \label{fig:testfig1}
\end{figure}



\begin{figure}[H]
  \centering
  \includegraphics[scale=0.35]{./Graphs/Figure1.6.png}
  \caption{Task1-Output}
  \label{fig:testfig1}
\end{figure}

The Figure 4 shows the result of the encrypted message. Further, we use the decrytion algorithm to double check the answer. The result matches to the original text.\\ 

\vspace*{-0.10cm}
\section{Decryption Implementation}

The decryption algorithm put the letters in columns by the key order '14023'; however, as we fill the columns by following the order of '20341'(Figure 5, left). Take '0' as an example, the '0' is filled up 1st in the RowMatrix's 2nd column. Then, it reads off the message by '01234' row by row(Figure 5, right). \\

\begin{figure}[H]
  \centering
  \includegraphics[scale=0.55]{./Graphs/Figure1.8.png}
  \caption{Task2-Decrypted Row Transposition Matrix \\}
  \label{fig:testfig1}
\end{figure}


\begin{figure}[H]
  \centering
  \includegraphics[scale=0.75]{./Graphs/Figure1.9.png}
  \caption{Task2-Decrypted Message}
  \label{fig:testfig1}
\end{figure}

The following is the pseudocode for the row transposition cipher algorithm : \\ 
\indent The first part finds the decoding sequence from the key 'NYITV'. Then, we write the encrypted message into a 2D RowTranspositionMatrix. Next, we use the matrix to record the rearrangement of the RowTranspositionMatrix. In the last step, we utilize the StringBuilder to build the encrypted message line by line through the RowMatrix.\\ \\ \\ \\ 


\vspace*{-1.5cm}

\floatname {algorithm} {\footnotesize Row Transposition Decryption Algorithm}
\renewcommand {\algorithmicrequire}{\textbf{input:}}
\renewcommand{\algorithmicensure}{\textbf{output:}}
\begin{algorithm}
  \caption{}\label{}
  \begin{algorithmic}[1]
  	\footnotesize
    \Require 'w' : Key and 'C' Encrypted plain-text
    \Ensure Decrypted plain-text
    \Function{rtcdecryption}{$w,C$}
      \State \footnotesize$ $
      \State $keylen \gets w.length()$
      \State $keyArray \gets key.toCharArray() $
      \State $messageArray \gets C.toCharArray() $
      \State $keyPosition \gets int [keylen]$
      \State \footnotesize$ $
	  \State $Sort\hspace{0.1cm} the \hspace{0.1cm}keyArray$  \Comment \tiny{sort the keyArray and assign it to a string}
	  \State\footnotesize $String s \gets String.valueOf(keyArray)$	  
	  \State \footnotesize$ $
	  \State $x \gets 0$ 
      \For{\texttt {each char c in dArray}} 
        \State  $keyPosition[x] \gets s.indexOf(c)$
        \State $Increament \hspace{0.1cm} x \hspace{0.1cm} by \hspace{0.1cm}1$	
      \EndFor
      \State \footnotesize$ $
      
      \State $cols \gets keylen$ 
      \State $rows \gets 0$
	  \If{$C's \hspace{0.1cm}length \hspace{0.1cm} mod \hspace{0.1cm} cols \hspace{0.1cm} equals \hspace{0.1cm} 0$} \Comment \tiny{calculate rows}
        \State\footnotesize $rows \gets C.length() / cols$
      \Else \Comment \tiny{Get the right order to decrypt the message in the matrix} \footnotesize
        \State \footnotesize$rows \gets C.length() / cols + 1$
     \EndIf
     
      \State \footnotesize$ $
      \State  $RowMatrix \gets char[rows][cols] $
       
 
	  \State $k \gets 0$
	  \For{\texttt{i to rows}}
	  	\For{\texttt{j to cols}}
        	\If{$count \hspace{0.1cm}k\hspace{0.1cm} equals\hspace{0.1cm} message \hspace{0.1cm} C's \hspace{0.1cm}length$}
        	   \State $while \hspace{0.1cm} k \hspace{0.1cm} equals \hspace{0.1cm} message's \hspace{0.1cm} length \hspace{0.1cm} and \hspace{0.1cm} j \hspace{0.1cm}   $
        	   \State $less \hspace{0.1cm}than \hspace{0.1cm} cols \hspace{0.1cm} keep \hspace{0.1cm}add \hspace{0.1cm}'X' \hspace{0.1cm} to \hspace{0.1cm}$
        	   \State $RowMatrix[i][j]$
        	   \State $break$
        	  
        	   \State $assign \hspace{0.1cm}  RowMatrix[j][keyPosition[i]] \hspace{0.1cm}  from$
        	   \State $messageArray[k]$
        	   \State \footnotesize $Increament\hspace{0.1cm} k\hspace{0.1cm} by \hspace{0.1cm}1$	
        	   \State \footnotesize$ $
		    \EndIf
        \EndFor  
      \EndFor  
               
      \State \footnotesize $ StringBuilder\hspace{0.1cm} str \gets StringBuilder()$
      \For{\texttt{i to rows}}
	  	\For{\texttt{j to cols}}
	  		\If{$RowMatrix[i][j] \hspace{0.1cm} unequal \hspace{0.1cm} to \hspace{0.1cm} 'X'$} 
	  		\State $ str.append(RowMatrix[i][j]) $
		    \EndIf
        \EndFor  
      \EndFor  
      
      \State \textbf{return} $str$\Comment \tiny{Decrypted message is str}
    \EndFunction
  \end{algorithmic}
\end{algorithm}


The following figure 7 shows the result of the decrypted plaintext. Further, we use the encryption algorithm to double-check the answer. The result matches the original encrypted message.

\begin{figure}[H]
  \centering
  \includegraphics[scale=0.35]{./Graphs/Figure2.0.png}
  \caption{Task2-Output}
  \label{fig:testfig1}
\end{figure}




\section{Conclusion}

With the growing use of computers and the internet, and an increasing need to transmit information quickly and securely, encryption through existing protocols (AES, RSA, 3DES, etc.) became essential. 
In the project, we can see that using only one round of encryption (row transportation) and a minor key (5 letters), the information is already quite challenging to decipher, and with the use of the protocols mentioned above that repeatedly (using transportation, substitution, and other resources), it becomes harder to decipher the messages. 
We also demonstrate in the project that the information is decrypted, just doing the inverse of the encryption procedure that needs to be done by the person who will receive the message.
 \\ 

%----------------------------------------------------------------------------------------
%	REFERENCE LIST
%----------------------------------------------------------------------------------------
\vspace*{5.5cm}
\begin{thebibliography}{} % Bibliography - this is intentionally simple in this template

\footnotesize[1]N. Hamza, “Row transposition ciphers - ppt download,” SlidePlayer. [Online]. Available: https://slideplayer.com/slide/13205094/. [Accessed: 28-Feb-2022].  \\ \\

\footnotesize[2]“Transposition cipher,” Wikipedia, 23-Feb-2022. [Online]. Available: https\:// en.wikipedia.org/wiki/Transposition\_cipher. [Accessed\: 28-Feb-2022]. \\ \\


 
\end{thebibliography}


%----------------------------------------------------------------------------------------

\end{document}
